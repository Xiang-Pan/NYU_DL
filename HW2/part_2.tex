\section{Implementation (50pts + 5pts extra credit)}

% \subsection{Warm Up}
% You need to fill all the ToDo places in the notebook ...
% Make sure you get the similar results as the result of autograd.

% \subsection{Implementation}

There are three notebooks in the practical part:
\begin{itemize}
    \item (25pts) Convolutional Neural Networks notebook:
    \href{https://drive.google.com/file/d/1ldhJS40UdIaXxdwfMdax6GtwAf8Srtwh/view?usp=sharing}{\texttt{hw2\_cnn.ipynb}}
    \item (20pts) Recurrent Neural Networks notebook:
    \href{https://drive.google.com/file/d/1DwNkLMIlz4YUk_ujai3tuUdD1N6-b7uX/view?usp=sharing}{\texttt{hw2\_rnn.ipynb}}
    \item (5pts + 5pts extra credit) :
        This builds on Section \ref{debug_loss} of the theoretical part.
        \begin{itemize}
            \item (5pts) Change the model training procedure of Section 8 in \\ \href{https://github.com/Atcold/pytorch-Deep-Learning/blob/master/08-seq_classification.ipynb}{\texttt{08-seq\_classification}} to make the training curves have no spikes. You should only change the training of the model, and not the model itself or the random seed.
            \item (5pts extra credit) Visualize the gradients and weights throughout training before and after you fix the training procedure.
        \end{itemize}
\end{itemize}

\textbf{Plase use your NYU Google Drive account to access the notebooks.} First two notebooks contain parts marked as \texttt{TODO}, where you should put your code. These notebooks are Google Colab notebooks, you should copy them to your drive, add your solutions, and then download and submit them to NYU Brightspace. The notebook from the class, if needed, can be uploaded to  Colab as well.